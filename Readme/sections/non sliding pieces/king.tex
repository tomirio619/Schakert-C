% !TeX spellcheck = en_US
\subsubsection{King}
%
\begin{figure}[H]
	\centering
	\newchessgame
	\setchessboard{showmover=false, smallboard}
	\chessboard[setpieces={Kd4},
		pgfstyle=straightmove,
		shortenstart=1ex,
		arrow=to,
		markmoves={
		d4-d5,
		d4-c5,
		d4-c4,
		d4-c3,
		d4-d3,
		d4-e3,
		d4-e4,
		d4-e5},
	]
	\captionof{figure}{All the king moves.}
	\label{fig:king moves}
\end{figure}
%
As you can see in \Cref{fig:king moves}, the king can move one square in each direction of our compass rose. 
Because the position of the king in the corresponding bitboard is one, we can use shifts on the bitboard to generate all spots the king can move to.
For each direction the king can move to, we specify the corresponding bit shift.
%
\begin{minted}{C++}
uint64_t getKingMoves(uint64_t king_loc, uint64_t friendly_pieces){
	uint64_t N  = northOne(king_loc);
	uint64_t NE = northEastOne(king_loc);
	uint64_t E  = eastOne(king_loc);
	uint64_t SE = southEastOne(king_loc);
	uint64_t S  = southOne(king_loc);
	uint64_t SW = southWestOne(king_loc);
	uint64_t W  = westOne(king_loc);
	uint64_t NW = northWestOne(king_loc);
	
	uint64_t king_moves = N | NE | E | SE | S | SW | W | NW;
	/*
	Final AND makes sure we only move to squares that are empty or occupied by an enemy piece.
	*/
	return king_moves & ~friendly_pieces;
}
\end{minted}
%
The implementation of the used methods can be seen in \Cref{fig: shifting abstractions}.
An important note is that we use the bitboard of the position of the king itself as a seed for the calculation as a ``mask'' for determining the movements.
\textbf{Masks} is data that is used in bitwise operations. As you can see, we use the OR operator where all of the generated possible moves represented as bitboards are involved.
By using the OR ($|$), we enable (setting the value of a bit to 1) all the bits to where the king could move.
In some cases, a bit shift could wrap a bit to another file that would clearly not be correct as a valid move for the king.
We demonstrate this `bit wrapping' by an example.
Assume the white king is on square \texttt{a3}.
The bitboard of the white king is shown in \Cref{fig: king org pos}.
%
\begin{figure}[H]
\centering
\subfloat[Original king position.]{
	\begin{tikzpicture}[baseline]
	\def\x{0.53}
	\draw[xstep=\x cm,ystep=\x cm,color=gray] (0,0) grid (\x*8,\x*8);
	\matrix[matrix of nodes,
	% http://tex.stackexchange.com/questions/15093/single-ampersand-used-with-wrong-catcode-error-using-tikz-matrix-in-beamer
	ampersand replacement=\&,
	inner sep=0pt,
	anchor=south west,
	nodes={inner sep=0pt,text width=\x cm,align=center,minimum height=\x cm}
	]{
		0 \& 0 \& 0 \& 0 \& 0 \& 0 \& 0 \& 0 \\
		0 \& 0 \& 0 \& 0 \& 0 \& 0 \& 0 \& 0 \\
		0 \& 0 \& 0 \& 0 \& 0 \& 0 \& 0 \& 0 \\
		0 \& 0 \& 0 \& 0 \& 0 \& 0 \& 0 \& 0 \\
		0 \& 0 \& 0 \& 0 \& 0 \& 0 \& 0 \& 0 \\
		\textbf{1} \& 0 \& 0 \& 0 \& 0 \& 0 \& 0 \& 0 \\
		0 \& 0 \& 0 \& 0 \& 0 \& 0 \& 0 \& 0 \\
		0 \& 0 \& 0 \& 0 \& 0 \& 0 \& 0 \& 0 \\
	};
	\end{tikzpicture}
	\label{fig: king org pos}
}
\hspace{1cm}
\subfloat[King position when calculating the move to the west.]{
	\begin{tikzpicture}[baseline]
	\def\x{0.53}
	\draw[xstep=\x cm,ystep=\x cm,color=gray] (0,0) grid (\x*8,\x*8);
	\matrix[matrix of nodes,
	% http://tex.stackexchange.com/questions/15093/single-ampersand-used-with-wrong-catcode-error-using-tikz-matrix-in-beamer
	ampersand replacement=\&,
	inner sep=0pt,
	anchor=south west,
	nodes={inner sep=0pt,text width=\x cm,align=center,minimum height=\x cm}
	]{
		0 \& 0 \& 0 \& 0 \& 0 \& 0 \& 0 \& 0 \\
		0 \& 0 \& 0 \& 0 \& 0 \& 0 \& 0 \& 0 \\
		0 \& 0 \& 0 \& 0 \& 0 \& 0 \& 0 \& 0 \\
		0 \& 0 \& 0 \& 0 \& 0 \& 0 \& 0 \& 0 \\
		0 \& 0 \& 0 \& 0 \& 0 \& 0 \& 0 \& 0 \\
		0 \& 0 \& 0 \& 0 \& 0 \& 0 \& 0 \& 0 \\
		0 \& 0 \& 0 \& 0 \& 0 \& 0 \& 0 \& \textbf{1} \\
		0 \& 0 \& 0 \& 0 \& 0 \& 0 \& 0 \& 0 \\
	};
	\end{tikzpicture}
	\label{fig:king pos after west shift}
}
\captionof{figure}{The position of the kig after shifting to the West when the king's original position is in \texttt{a} file.}
\end{figure}
%
Lets try to generate the move to the west direction, which essentially is a bit shift of 1 to the LSB, or a right bit shift.
We end up with the bitboard shown in \Cref{fig:king pos after west shift}.
But this position is not reachable at all for the king from its original position!
To fix this kind of mistakes, bit masks come into play.
When we want to calculate the NW, W and SW positions for a king that is standing in the \texttt{a} file, we know for sure that they will produce invalid bitboards representing king moves.
To prevent this, we use bit masks.
If the king is in the \texttt{a} file, we perform the bit shift and clear the \texttt{h} file afterwards.
The mask for clipping out the \texttt{h} file is as follows:
%
\begin{figure}[H]
\centering
\begin{tikzpicture}[baseline]
\def\x{0.53}
\draw[xstep=\x cm,ystep=\x cm,color=gray] (0,0) grid (\x*8,\x*8);
\matrix[matrix of nodes,
% http://tex.stackexchange.com/questions/15093/single-ampersand-used-with-wrong-catcode-error-using-tikz-matrix-in-beamer
ampersand replacement=\&,
inner sep=0pt,
anchor=south west,
nodes={inner sep=0pt,text width=\x cm,align=center,minimum height=\x cm}
]{
	1 \& 1 \& 1 \& 1 \& 1 \& 1 \& 1 \& 0 \\
	1 \& 1 \& 1 \& 1 \& 1 \& 1 \& 1 \& 0 \\
	1 \& 1 \& 1 \& 1 \& 1 \& 1 \& 1 \& 0 \\
	1 \& 1 \& 1 \& 1 \& 1 \& 1 \& 1 \& 0 \\
	1 \& 1 \& 1 \& 1 \& 1 \& 1 \& 1 \& 0 \\
	1 \& 1 \& 1 \& 1 \& 1 \& 1 \& 1 \& 0 \\
	1 \& 1 \& 1 \& 1 \& 1 \& 1 \& 1 \& 0 \\
	1 \& 1 \& 1 \& 1 \& 1 \& 1 \& 1 \& 0 \\
};
\end{tikzpicture}
\captionof{figure}{Bit mask for clipping out the \texttt{h} file.}
\end{figure}
%
\begin{figure}
	\centering
	\subfloat[The original bitboard of the King]{
	\begin{tikzpicture}[baseline]
	\def\x{0.53}
	\draw[xstep=\x cm,ystep=\x cm,color=gray] (0,0) grid (\x*8,\x*8);
	\matrix[matrix of nodes,
	% http://tex.stackexchange.com/questions/15093/single-ampersand-used-with-wrong-catcode-error-using-tikz-matrix-in-beamer
	ampersand replacement=\&,
	inner sep=0pt,
	anchor=south west,
	nodes={inner sep=0pt,text width=\x cm,align=center,minimum height=\x cm}
	]{
		0 \& 0 \& 0 \& 0 \& 0 \& 0 \& 0 \& 0 \\
		0 \& 0 \& 0 \& 0 \& 0 \& 0 \& 0 \& 0 \\
		0 \& 0 \& 0 \& 0 \& 0 \& 0 \& 0 \& 0 \\
		0 \& 0 \& 0 \& 0 \& 0 \& 0 \& 0 \& 0 \\
		0 \& 0 \& 0 \& 0 \& 0 \& 0 \& 0 \& 0 \\
		\textbf{1} \& 0 \& 0 \& 0 \& 0 \& 0 \& 0 \& 0 \\
		0 \& 0 \& 0 \& 0 \& 0 \& 0 \& 0 \& 0 \\
		0 \& 0 \& 0 \& 0 \& 0 \& 0 \& 0 \& 0 \\
	};
	\end{tikzpicture}
	}
	\hspace{2cm}
	\subfloat[The bitboard of the king after shifting to the west.]{
	\begin{tikzpicture}[baseline]
	\def\x{0.53}
	\draw[xstep=\x cm,ystep=\x cm,color=gray] (0,0) grid (\x*8,\x*8);
	\matrix[matrix of nodes,
	% http://tex.stackexchange.com/questions/15093/single-ampersand-used-with-wrong-catcode-error-using-tikz-matrix-in-beamer
	ampersand replacement=\&,
	inner sep=0pt,
	anchor=south west,
	nodes={inner sep=0pt,text width=\x cm,align=center,minimum height=\x cm}
	]{
		0 \& 0 \& 0 \& 0 \& 0 \& 0 \& 0 \& 0 \\
		0 \& 0 \& 0 \& 0 \& 0 \& 0 \& 0 \& 0 \\
		0 \& 0 \& 0 \& 0 \& 0 \& 0 \& 0 \& 0 \\
		0 \& 0 \& 0 \& 0 \& 0 \& 0 \& 0 \& 0 \\
		0 \& 0 \& 0 \& 0 \& 0 \& 0 \& 0 \& 0 \\
		0 \& 0 \& 0 \& 0 \& 0 \& 0 \& 0 \& 0 \\
		0 \& 0 \& 0 \& 0 \& 0 \& 0 \& 0 \& \textbf{1} \\
		0 \& 0 \& 0 \& 0 \& 0 \& 0 \& 0 \& 0 \\
	};
	\end{tikzpicture}
	}
	\vfill
	\subfloat[The bitboard for clearing the \texttt{h} file]{
	\begin{tikzpicture}[baseline]
		\def\x{0.53}
		\draw[xstep=\x cm,ystep=\x cm,color=gray] (0,0) grid (\x*8,\x*8);
		\matrix[matrix of nodes,
		% http://tex.stackexchange.com/questions/15093/single-ampersand-used-with-wrong-catcode-error-using-tikz-matrix-in-beamer
		ampersand replacement=\&,
		inner sep=0pt,
		anchor=south west,
		nodes={inner sep=0pt,text width=\x cm,align=center,minimum height=\x cm}
		]{
		1 \& 1 \& 1 \& 1 \& 1 \& 1 \& 1 \& 0 \\
		1 \& 1 \& 1 \& 1 \& 1 \& 1 \& 1 \& 0 \\
		1 \& 1 \& 1 \& 1 \& 1 \& 1 \& 1 \& 0 \\
		1 \& 1 \& 1 \& 1 \& 1 \& 1 \& 1 \& 0 \\
		1 \& 1 \& 1 \& 1 \& 1 \& 1 \& 1 \& 0 \\
		1 \& 1 \& 1 \& 1 \& 1 \& 1 \& 1 \& 0 \\
		1 \& 1 \& 1 \& 1 \& 1 \& 1 \& 1 \& 0 \\
		1 \& 1 \& 1 \& 1 \& 1 \& 1 \& 1 \& 0 \\
		};
		\end{tikzpicture}
	}
	\hspace{2cm}
	\subfloat[The bitboard of the king after using the mask that clears the \texttt{h} file]{
		\begin{tikzpicture}[baseline]
		\def\x{0.53}
		\draw[xstep=\x cm,ystep=\x cm,color=gray] (0,0) grid (\x*8,\x*8);
		\matrix[matrix of nodes,
		% http://tex.stackexchange.com/questions/15093/single-ampersand-used-with-wrong-catcode-error-using-tikz-matrix-in-beamer
		ampersand replacement=\&,
		inner sep=0pt,
		anchor=south west,
		nodes={inner sep=0pt,text width=\x cm,align=center,minimum height=\x cm}
		]{
			0 \& 0 \& 0 \& 0 \& 0 \& 0 \& 0 \& 0 \\
			0 \& 0 \& 0 \& 0 \& 0 \& 0 \& 0 \& 0 \\
			0 \& 0 \& 0 \& 0 \& 0 \& 0 \& 0 \& 0 \\
			0 \& 0 \& 0 \& 0 \& 0 \& 0 \& 0 \& 0 \\
			0 \& 0 \& 0 \& 0 \& 0 \& 0 \& 0 \& 0 \\
			0 \& 0 \& 0 \& 0 \& 0 \& 0 \& 0 \& 0 \\
			0 \& 0 \& 0 \& 0 \& 0 \& 0 \& 0 \& 0 \\
			0 \& 0 \& 0 \& 0 \& 0 \& 0 \& 0 \& 0 \\
		};
		\end{tikzpicture}
		}
	\captionof{figure}{Preventing wrong bitboards in king move generation, when the king is in \texttt{a} file and we generate a move in the west direction. This is done by clipping out the \texttt{h} file using a bit mask. This mask is applied after the bit shift (these masks are called post-shift masks).}
\end{figure}
%
Note that it is not possible for a king to stand in rank 1, to end up in rank 8 after move generation of N, NE or NW. 
Because the bit will fall of the edge (bit overflow), the resulting bitboard will always contain zero's only.
The same applies for a king standing on rank 8 and generating S, SE or SW moves, which are corrected by an bit underflow.
In the same line as described for a kin in \texttt{a} file, we also have to use a bit mask in \texttt{h} file for directions.
In summary:
%
\begin{itemize}
	\item For the move generation in direction N and W, we do not have to perform a correcting clipping mask;
	
	\item For the move generations in direction E, NE and SE, we have to perform a correcting clipping mask: clipping out the \texttt{a} file.
	
	\item For the move generations in direction W, NW, SW, we have to perform a correcting clipping mask:
	clipping out the \texttt{h} file.
\end{itemize}
%
The implementation of the bit shifts in \Cref{fig: shifting abstractions} removes the possible wrongly wrapped bits. Take a look at them!
Note that we can also improve the move generation for the king.
If we OR the bitboards representing the east and west moves with the original position of the king,
we get a bitboard we can use to get the moves in the other directions by two shifts:
%
\begin{itemize}
	\item Shifting up one rank to determine the NW, N and NE moves.
	\item Shifting down one rank to determine the SE, S and SW moves.
\end{itemize}
%
This is faster then our previous method, which involves calculating all of the 8 directions independently.
Make sure to use the bit mask of all the friendly pieces in the end, otherwise, the current position of the king will also be seen as a valid new position.