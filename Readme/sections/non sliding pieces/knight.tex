% !TeX spellcheck = en_US
\subsubsection{Knight}
%
\begin{figure}[H]
	\centering
	\newchessgame
	\setchessboard{showmover=false, smallboard}
	\chessboard[setpieces={Nd4},
		pgfstyle=knightmove,
		shortenstart=1ex,
		arrow=to,
		markmoves={
		d4-c6,
		d4-b5,
		d4-b3,
		d4-c2,
		d4-e2,
		d4-f3,
		d4-f5,
		d4-e6},
	]
	\captionof{figure}{All the knight moves.}
	\label{fig:knight moves}
\end{figure}
%
In \Cref{fig:knight moves}, you can see the possible moves of the knight.
Looking at the file and row distance, it could to files and rows that are a distance 2 away from their current row or file.
In contrary to the masks applied to our king, we need to apply either a mask that clips both the \texttt{a} and \texttt{b} file, or the \texttt{g} and \texttt{h}. Again, we do not have to worry about mask clipping, because the integer overflows and underflows will take care of that for us.
The custom wind rose of the knight positions is shown in \Cref{fig: knight wind rose}.
%
\begin{figure}[H]
\begin{adjustwidth}{.35\textwidth}{}
\begin{minted}{raw}
. . . NNW . . . NNE . . . .
. . . . +15 . +17 . . . . .
. . . . | . . . | . . . . .
NWW +6__| . . . |__+10 NEE.
. . . . \ . . ./. . . . . .
. . . . . >0< . . . . . . .
. .  __ / . . .\ __ . . . .
SWW -10 | . . . | . -6 SEE.
. . . . | . . . | . . . . .
. . . -17 . . -15 . . . . .
. . . SWW . . SEE . . . . .
\end{minted}
\end{adjustwidth}
\captionof{figure}{Custom wind rose for the possible knight moves.}
\label{fig: knight wind rose}
\end{figure}
%
Now the following rules apply:
%
\begin{itemize}
	\item We have to clip the \texttt{h} file if we calculate spot NNW.
	\item We have to clip the \texttt{h} file if we calculate spot SSW.
	\item We have to clip the \texttt{a} file if we calculate spot NNE.
	\item We have to clip the \texttt{a} file if we calculate spot SSE.
	\item We have to clip both the \texttt{h} file and the \texttt{g} file if we calculate spot NWW.
	\item We have to clip both the \texttt{h} file and the \texttt{g} file if we calculate spot SWW.
	\item We have to clip both the \texttt{a} file and the \texttt{b} file if we calculate spot NEE.
	\item We have to clip both the \texttt{a} file and the \texttt{b} file if we calculate spot SEE.
\end{itemize}
%
This gives code shown in \Cref{fig: knight moves}.
%
\begin{listing}
\begin{minted}{C++}
uint64_t getKnightMoves(uint64_t knight_loc, uint64_t friendly_pieces) {
	uint64_t NNW_clip = clear_file[FILE_H];
	uint64_t SSW_clip = clear_file[FILE_H];
	uint64_t NNE_clip = clear_file[FILE_A];
	uint64_t SSE_clip = clear_file[FILE_A];
	
	uint64_t NWW_clip = clear_file[FILE_H] & clear_file[FILE_G];
	uint64_t SWW_clip = clear_file[FILE_H] & clear_file[FILE_G];
	uint64_t NEE_clip = clear_file[FILE_A] & clear_file[FILE_B];
	uint64_t SEE_clip = clear_file[FILE_A] & clear_file[FILE_B];
	
	uint64_t NWW = (knight_loc & NWW_clip) << 6;
	uint64_t NEE = (knight_loc & NEE_clip) << 10;
	uint64_t NNW = (knight_loc & NNW_clip) << 15;
	uint64_t NNE = (knight_loc & NNE_clip) << 17;
	
	uint64_t SEE = (knight_loc & SEE_clip) >> 6;
	uint64_t SWW = (knight_loc & SWW_clip) >> 10;
	uint64_t SSE = (knight_loc & SSW_clip) >> 15;
	uint64_t SSW = (knight_loc & SSW_clip) >> 17;
	
	uint64_t knight_moves = NWW | NEE | NNW | NNE | SEE | SWW | SSW | SSE;
	return knight_moves & ~friendly_pieces;
}
\end{minted}
\captionof{listing}{Calculation of the the knight moveset.}
\label{fig: knight moves}
\end{listing}
%