% !TeX spellcheck = en_US
\section{Move Generation}
The non sliding pieces on a chess board are as follows: King, Knight and Pawn. 
The algorithm for generating moves for non sliding pieces are easier to understand and compute than for sliding pieces.
Therefore, we decided to take a look at these pieces first.
At this point, we only look at places where a piece can move (non-capture moves and capture moves).
As you know, a move is not valid if it puts its own king in check.
In this section, we do not take this rule into account.
As we will also be talking in the direction a piece can move to, we rely on the compass rose shown in \Cref{fig: compass rose}.
%
\begin{figure}
\begin{adjustwidth}{.37\textwidth}{}
\begin{minted}{raw}
. NW. . . N . . .NE . .
. . . . . . . . . . . .
. . +7. . +8. . .+9 . . 
. . . . \ .|. / . . . .
W . -1. <-.0. ->.+1 . E
. . . . / .|. \ . . . .
. . -9. . -8 . .-7. . .
. . . . . . . . . . . .
. SW. . . S . . .SE . .
\end{minted}
\end{adjustwidth}
\captionof{figure}{Our compass rose.}
\label{fig: compass rose}
\end{figure}
%
For better readability, we use abstractions for shifting into a direction one time.
This can be seen in \Cref{fig: shifting abstractions}.
Do not worry if you do not understand the operations that are shown that code block.
It will be explained when we are going to find the move set for the king, which follows directly after this section.
%
\begin{listing}[H]
\begin{minted}{c++}
/*
These are post-shift masks. 
This means that we first apply the bitshift,
and then remove unwanted wraps that could occur
in certain circumstances.
*/

uint64_t northOne(uint64_t bitboard){
	return bitboard << 8;
}

uint64_t southOne(uint64_t bitboard){
	return bitboard >> 8;
}

uint64_t eastOne(uint64_t bitboard){
	return (bitboard << 1) & Utils.ClearFile(FILE_A);
}

uint64_t southEastOne(uint64_t bitboard){
	return (bitboard >> 7) & Utils.ClearFile(FILE_A);
}

uint64_t northEastOne(uint64_t bitboard){
	return (bitboard << 9) & Utils.ClearFile(FILE_A);
}

uint64_t westOne(uint64_t bitboard){
	return (bitboard >> 1) & Utils.ClearFile(FILE_H);
}

uint64_t northWestOne(uint64_t bitboard){
	return (bitboard << 7) & Utils.ClearFile(FILE_H);
}

uint64_t southWestOne(uint64_t bitboard){
	return (bitboard >> 9) & Utils.ClearFile(FILE_H);
}
\end{minted}
\captionof{listing}{Post-shift masks for removing unwanted wraps when shifting a bitboard to obtain a move.}
\label{fig: shifting abstractions}
\end{listing}
%
\subsection{Non Sliding Pieces}
% !TeX spellcheck = en_US
\subsubsection{King}
%
\begin{figure}[H]
	\centering
	\newchessgame
	\setchessboard{showmover=false, smallboard}
	\chessboard[setpieces={Kd4},
		pgfstyle=straightmove,
		shortenstart=1ex,
		arrow=to,
		markmoves={
		d4-d5,
		d4-c5,
		d4-c4,
		d4-c3,
		d4-d3,
		d4-e3,
		d4-e4,
		d4-e5},
	]
	\captionof{figure}{All the king moves.}
	\label{fig:king moves}
\end{figure}
%
As you can see in \Cref{fig:king moves}, the king can move one square in each direction of our compass rose. 
Because the position of the king in the corresponding bitboard is one, we can use shifts on the bitboard to generate all spots the king can move to.
For each direction the king can move to, we specify the corresponding bit shift.
%
\begin{minted}{C++}
uint64_t getKingMoves(uint64_t king_loc, uint64_t friendly_pieces){
	uint64_t N  = northOne(king_loc);
	uint64_t NE = northEastOne(king_loc);
	uint64_t E  = eastOne(king_loc);
	uint64_t SE = southEastOne(king_loc);
	uint64_t S  = southOne(king_loc);
	uint64_t SW = southWestOne(king_loc);
	uint64_t W  = westOne(king_loc);
	uint64_t NW = northWestOne(king_loc);
	
	uint64_t king_moves = N | NE | E | SE | S | SW | W | NW;
	/*
	Final AND makes sure we only move to squares that are empty or occupied by an enemy piece.
	*/
	return king_moves & ~friendly_pieces;
}
\end{minted}
%
The implementation of the used methods can be seen in \Cref{fig: shifting abstractions}.
An important note is that we use the bitboard of the position of the king itself as a seed for the calculation as a ``mask'' for determining the movements.
\textbf{Masks} is data that is used in bitwise operations. As you can see, we use the OR operator where all of the generated possible moves represented as bitboards are involved.
By using the OR ($|$), we enable (setting the value of a bit to 1) all the bits to where the king could move.
In some cases, a bit shift could wrap a bit to another file that would clearly not be correct as a valid move for the king.
We demonstrate this `bit wrapping' by an example.
Assume the white king is on square \texttt{a3}.
The bitboard of the white king is shown in \Cref{fig: king org pos}.
%
\begin{figure}[H]
\centering
\subfloat[Original king position.]{
	\begin{tikzpicture}[baseline]
	\def\x{0.53}
	\draw[xstep=\x cm,ystep=\x cm,color=gray] (0,0) grid (\x*8,\x*8);
	\matrix[matrix of nodes,
	% http://tex.stackexchange.com/questions/15093/single-ampersand-used-with-wrong-catcode-error-using-tikz-matrix-in-beamer
	ampersand replacement=\&,
	inner sep=0pt,
	anchor=south west,
	nodes={inner sep=0pt,text width=\x cm,align=center,minimum height=\x cm}
	]{
		0 \& 0 \& 0 \& 0 \& 0 \& 0 \& 0 \& 0 \\
		0 \& 0 \& 0 \& 0 \& 0 \& 0 \& 0 \& 0 \\
		0 \& 0 \& 0 \& 0 \& 0 \& 0 \& 0 \& 0 \\
		0 \& 0 \& 0 \& 0 \& 0 \& 0 \& 0 \& 0 \\
		0 \& 0 \& 0 \& 0 \& 0 \& 0 \& 0 \& 0 \\
		\textbf{1} \& 0 \& 0 \& 0 \& 0 \& 0 \& 0 \& 0 \\
		0 \& 0 \& 0 \& 0 \& 0 \& 0 \& 0 \& 0 \\
		0 \& 0 \& 0 \& 0 \& 0 \& 0 \& 0 \& 0 \\
	};
	\end{tikzpicture}
	\label{fig: king org pos}
}
\hspace{1cm}
\subfloat[King position when calculating the move to the west.]{
	\begin{tikzpicture}[baseline]
	\def\x{0.53}
	\draw[xstep=\x cm,ystep=\x cm,color=gray] (0,0) grid (\x*8,\x*8);
	\matrix[matrix of nodes,
	% http://tex.stackexchange.com/questions/15093/single-ampersand-used-with-wrong-catcode-error-using-tikz-matrix-in-beamer
	ampersand replacement=\&,
	inner sep=0pt,
	anchor=south west,
	nodes={inner sep=0pt,text width=\x cm,align=center,minimum height=\x cm}
	]{
		0 \& 0 \& 0 \& 0 \& 0 \& 0 \& 0 \& 0 \\
		0 \& 0 \& 0 \& 0 \& 0 \& 0 \& 0 \& 0 \\
		0 \& 0 \& 0 \& 0 \& 0 \& 0 \& 0 \& 0 \\
		0 \& 0 \& 0 \& 0 \& 0 \& 0 \& 0 \& 0 \\
		0 \& 0 \& 0 \& 0 \& 0 \& 0 \& 0 \& 0 \\
		0 \& 0 \& 0 \& 0 \& 0 \& 0 \& 0 \& 0 \\
		0 \& 0 \& 0 \& 0 \& 0 \& 0 \& 0 \& \textbf{1} \\
		0 \& 0 \& 0 \& 0 \& 0 \& 0 \& 0 \& 0 \\
	};
	\end{tikzpicture}
	\label{fig:king pos after west shift}
}
\captionof{figure}{The position of the kig after shifting to the West when the king's original position is in \texttt{a} file.}
\end{figure}
%
Lets try to generate the move to the west direction, which essentially is a bit shift of 1 to the LSB, or a right bit shift.
We end up with the bitboard shown in \Cref{fig:king pos after west shift}.
But this position is not reachable at all for the king from its original position!
To fix this kind of mistakes, bit masks come into play.
When we want to calculate the NW, W and SW positions for a king that is standing in the \texttt{a} file, we know for sure that they will produce invalid bitboards representing king moves.
To prevent this, we use bit masks.
If the king is in the \texttt{a} file, we perform the bit shift and clear the \texttt{h} file afterwards.
The mask for clipping out the \texttt{h} file is as follows:
%
\begin{figure}[H]
\centering
\begin{tikzpicture}[baseline]
\def\x{0.53}
\draw[xstep=\x cm,ystep=\x cm,color=gray] (0,0) grid (\x*8,\x*8);
\matrix[matrix of nodes,
% http://tex.stackexchange.com/questions/15093/single-ampersand-used-with-wrong-catcode-error-using-tikz-matrix-in-beamer
ampersand replacement=\&,
inner sep=0pt,
anchor=south west,
nodes={inner sep=0pt,text width=\x cm,align=center,minimum height=\x cm}
]{
	1 \& 1 \& 1 \& 1 \& 1 \& 1 \& 1 \& 0 \\
	1 \& 1 \& 1 \& 1 \& 1 \& 1 \& 1 \& 0 \\
	1 \& 1 \& 1 \& 1 \& 1 \& 1 \& 1 \& 0 \\
	1 \& 1 \& 1 \& 1 \& 1 \& 1 \& 1 \& 0 \\
	1 \& 1 \& 1 \& 1 \& 1 \& 1 \& 1 \& 0 \\
	1 \& 1 \& 1 \& 1 \& 1 \& 1 \& 1 \& 0 \\
	1 \& 1 \& 1 \& 1 \& 1 \& 1 \& 1 \& 0 \\
	1 \& 1 \& 1 \& 1 \& 1 \& 1 \& 1 \& 0 \\
};
\end{tikzpicture}
\captionof{figure}{Bit mask for clipping out the \texttt{h} file.}
\end{figure}
%
\begin{figure}
	\centering
	\subfloat[The original bitboard of the King]{
	\begin{tikzpicture}[baseline]
	\def\x{0.53}
	\draw[xstep=\x cm,ystep=\x cm,color=gray] (0,0) grid (\x*8,\x*8);
	\matrix[matrix of nodes,
	% http://tex.stackexchange.com/questions/15093/single-ampersand-used-with-wrong-catcode-error-using-tikz-matrix-in-beamer
	ampersand replacement=\&,
	inner sep=0pt,
	anchor=south west,
	nodes={inner sep=0pt,text width=\x cm,align=center,minimum height=\x cm}
	]{
		0 \& 0 \& 0 \& 0 \& 0 \& 0 \& 0 \& 0 \\
		0 \& 0 \& 0 \& 0 \& 0 \& 0 \& 0 \& 0 \\
		0 \& 0 \& 0 \& 0 \& 0 \& 0 \& 0 \& 0 \\
		0 \& 0 \& 0 \& 0 \& 0 \& 0 \& 0 \& 0 \\
		0 \& 0 \& 0 \& 0 \& 0 \& 0 \& 0 \& 0 \\
		\textbf{1} \& 0 \& 0 \& 0 \& 0 \& 0 \& 0 \& 0 \\
		0 \& 0 \& 0 \& 0 \& 0 \& 0 \& 0 \& 0 \\
		0 \& 0 \& 0 \& 0 \& 0 \& 0 \& 0 \& 0 \\
	};
	\end{tikzpicture}
	}
	\hspace{2cm}
	\subfloat[The bitboard of the king after shifting to the west.]{
	\begin{tikzpicture}[baseline]
	\def\x{0.53}
	\draw[xstep=\x cm,ystep=\x cm,color=gray] (0,0) grid (\x*8,\x*8);
	\matrix[matrix of nodes,
	% http://tex.stackexchange.com/questions/15093/single-ampersand-used-with-wrong-catcode-error-using-tikz-matrix-in-beamer
	ampersand replacement=\&,
	inner sep=0pt,
	anchor=south west,
	nodes={inner sep=0pt,text width=\x cm,align=center,minimum height=\x cm}
	]{
		0 \& 0 \& 0 \& 0 \& 0 \& 0 \& 0 \& 0 \\
		0 \& 0 \& 0 \& 0 \& 0 \& 0 \& 0 \& 0 \\
		0 \& 0 \& 0 \& 0 \& 0 \& 0 \& 0 \& 0 \\
		0 \& 0 \& 0 \& 0 \& 0 \& 0 \& 0 \& 0 \\
		0 \& 0 \& 0 \& 0 \& 0 \& 0 \& 0 \& 0 \\
		0 \& 0 \& 0 \& 0 \& 0 \& 0 \& 0 \& 0 \\
		0 \& 0 \& 0 \& 0 \& 0 \& 0 \& 0 \& \textbf{1} \\
		0 \& 0 \& 0 \& 0 \& 0 \& 0 \& 0 \& 0 \\
	};
	\end{tikzpicture}
	}
	\vfill
	\subfloat[The bitboard for clearing the \texttt{h} file]{
	\begin{tikzpicture}[baseline]
		\def\x{0.53}
		\draw[xstep=\x cm,ystep=\x cm,color=gray] (0,0) grid (\x*8,\x*8);
		\matrix[matrix of nodes,
		% http://tex.stackexchange.com/questions/15093/single-ampersand-used-with-wrong-catcode-error-using-tikz-matrix-in-beamer
		ampersand replacement=\&,
		inner sep=0pt,
		anchor=south west,
		nodes={inner sep=0pt,text width=\x cm,align=center,minimum height=\x cm}
		]{
		1 \& 1 \& 1 \& 1 \& 1 \& 1 \& 1 \& 0 \\
		1 \& 1 \& 1 \& 1 \& 1 \& 1 \& 1 \& 0 \\
		1 \& 1 \& 1 \& 1 \& 1 \& 1 \& 1 \& 0 \\
		1 \& 1 \& 1 \& 1 \& 1 \& 1 \& 1 \& 0 \\
		1 \& 1 \& 1 \& 1 \& 1 \& 1 \& 1 \& 0 \\
		1 \& 1 \& 1 \& 1 \& 1 \& 1 \& 1 \& 0 \\
		1 \& 1 \& 1 \& 1 \& 1 \& 1 \& 1 \& 0 \\
		1 \& 1 \& 1 \& 1 \& 1 \& 1 \& 1 \& 0 \\
		};
		\end{tikzpicture}
	}
	\hspace{2cm}
	\subfloat[The bitboard of the king after using the mask that clears the \texttt{h} file]{
		\begin{tikzpicture}[baseline]
		\def\x{0.53}
		\draw[xstep=\x cm,ystep=\x cm,color=gray] (0,0) grid (\x*8,\x*8);
		\matrix[matrix of nodes,
		% http://tex.stackexchange.com/questions/15093/single-ampersand-used-with-wrong-catcode-error-using-tikz-matrix-in-beamer
		ampersand replacement=\&,
		inner sep=0pt,
		anchor=south west,
		nodes={inner sep=0pt,text width=\x cm,align=center,minimum height=\x cm}
		]{
			0 \& 0 \& 0 \& 0 \& 0 \& 0 \& 0 \& 0 \\
			0 \& 0 \& 0 \& 0 \& 0 \& 0 \& 0 \& 0 \\
			0 \& 0 \& 0 \& 0 \& 0 \& 0 \& 0 \& 0 \\
			0 \& 0 \& 0 \& 0 \& 0 \& 0 \& 0 \& 0 \\
			0 \& 0 \& 0 \& 0 \& 0 \& 0 \& 0 \& 0 \\
			0 \& 0 \& 0 \& 0 \& 0 \& 0 \& 0 \& 0 \\
			0 \& 0 \& 0 \& 0 \& 0 \& 0 \& 0 \& 0 \\
			0 \& 0 \& 0 \& 0 \& 0 \& 0 \& 0 \& 0 \\
		};
		\end{tikzpicture}
		}
	\captionof{figure}{Preventing wrong bitboards in king move generation, when the king is in \texttt{a} file and we generate a move in the west direction. This is done by clipping out the \texttt{h} file using a bit mask. This mask is applied after the bit shift (these masks are called post-shift masks).}
\end{figure}
%
Note that it is not possible for a king to stand in rank 1, to end up in rank 8 after move generation of N, NE or NW. 
Because the bit will fall of the edge (bit overflow), the resulting bitboard will always contain zero's only.
The same applies for a king standing on rank 8 and generating S, SE or SW moves, which are corrected by an bit underflow.
In the same line as described for a kin in \texttt{a} file, we also have to use a bit mask in \texttt{h} file for directions.
In summary:
%
\begin{itemize}
	\item For the move generation in direction N and W, we do not have to perform a correcting clipping mask;
	
	\item For the move generations in direction E, NE and SE, we have to perform a correcting clipping mask: clipping out the \texttt{a} file.
	
	\item For the move generations in direction W, NW, SW, we have to perform a correcting clipping mask:
	clipping out the \texttt{h} file.
\end{itemize}
%
The implementation of the bit shifts in \Cref{fig: shifting abstractions} removes the possible wrongly wrapped bits. Take a look at them!
Note that we can also improve the move generation for the king.
If we OR the bitboards representing the east and west moves with the original position of the king,
we get a bitboard we can use to get the moves in the other directions by two shifts:
%
\begin{itemize}
	\item Shifting up one rank to determine the NW, N and NE moves.
	\item Shifting down one rank to determine the SE, S and SW moves.
\end{itemize}
%
This is faster then our previous method, which involves calculating all of the 8 directions independently.
Make sure to use the bit mask of all the friendly pieces in the end, otherwise, the current position of the king will also be seen as a valid new position.
% !TeX spellcheck = en_US
\subsubsection{Knight}
%
\begin{figure}[H]
	\centering
	\newchessgame
	\setchessboard{showmover=false, smallboard}
	\chessboard[setpieces={Nd4},
		pgfstyle=knightmove,
		shortenstart=1ex,
		arrow=to,
		markmoves={
		d4-c6,
		d4-b5,
		d4-b3,
		d4-c2,
		d4-e2,
		d4-f3,
		d4-f5,
		d4-e6},
	]
	\captionof{figure}{All the knight moves.}
	\label{fig:knight moves}
\end{figure}
%
In \Cref{fig:knight moves}, you can see the possible moves of the knight.
Looking at the file and row distance, it could to files and rows that are a distance 2 away from their current row or file.
In contrary to the masks applied to our king, we need to apply either a mask that clips both the \texttt{a} and \texttt{b} file, or the \texttt{g} and \texttt{h}. Again, we do not have to worry about mask clipping, because the integer overflows and underflows will take care of that for us.
The custom wind rose of the knight positions is shown in \Cref{fig: knight wind rose}.
%
\begin{figure}[H]
\begin{adjustwidth}{.35\textwidth}{}
\begin{minted}{raw}
. . . NNW . . . NNE . . . .
. . . . +15 . +17 . . . . .
. . . . | . . . | . . . . .
NWW +6__| . . . |__+10 NEE.
. . . . \ . . ./. . . . . .
. . . . . >0< . . . . . . .
. .  __ / . . .\ __ . . . .
SWW -10 | . . . | . -6 SEE.
. . . . | . . . | . . . . .
. . . -17 . . -15 . . . . .
. . . SWW . . SEE . . . . .
\end{minted}
\end{adjustwidth}
\captionof{figure}{Custom wind rose for the possible knight moves.}
\label{fig: knight wind rose}
\end{figure}
%
Now the following rules apply:
%
\begin{itemize}
	\item We have to clip the \texttt{h} file if we calculate spot NNW.
	\item We have to clip the \texttt{h} file if we calculate spot SSW.
	\item We have to clip the \texttt{a} file if we calculate spot NNE.
	\item We have to clip the \texttt{a} file if we calculate spot SSE.
	\item We have to clip both the \texttt{h} file and the \texttt{g} file if we calculate spot NWW.
	\item We have to clip both the \texttt{h} file and the \texttt{g} file if we calculate spot SWW.
	\item We have to clip both the \texttt{a} file and the \texttt{b} file if we calculate spot NEE.
	\item We have to clip both the \texttt{a} file and the \texttt{b} file if we calculate spot SEE.
\end{itemize}
%
This gives code shown in \Cref{fig: knight moves}.
%
\begin{listing}
\begin{minted}{C++}
uint64_t getKnightMoves(uint64_t knight_loc, uint64_t friendly_pieces) {
	uint64_t NNW_clip = clear_file[FILE_H];
	uint64_t SSW_clip = clear_file[FILE_H];
	uint64_t NNE_clip = clear_file[FILE_A];
	uint64_t SSE_clip = clear_file[FILE_A];
	
	uint64_t NWW_clip = clear_file[FILE_H] & clear_file[FILE_G];
	uint64_t SWW_clip = clear_file[FILE_H] & clear_file[FILE_G];
	uint64_t NEE_clip = clear_file[FILE_A] & clear_file[FILE_B];
	uint64_t SEE_clip = clear_file[FILE_A] & clear_file[FILE_B];
	
	uint64_t NWW = (knight_loc & NWW_clip) << 6;
	uint64_t NEE = (knight_loc & NEE_clip) << 10;
	uint64_t NNW = (knight_loc & NNW_clip) << 15;
	uint64_t NNE = (knight_loc & NNE_clip) << 17;
	
	uint64_t SEE = (knight_loc & SEE_clip) >> 6;
	uint64_t SWW = (knight_loc & SWW_clip) >> 10;
	uint64_t SSE = (knight_loc & SSW_clip) >> 15;
	uint64_t SSW = (knight_loc & SSW_clip) >> 17;
	
	uint64_t knight_moves = NWW | NEE | NNW | NNE | SEE | SWW | SSW | SSE;
	return knight_moves & ~friendly_pieces;
}
\end{minted}
\captionof{listing}{Calculation of the the knight moveset.}
\label{fig: knight moves}
\end{listing}
%
% !TeX spellcheck = en_US
\subsubsection{Pawn}
\begin{figure}[H]
	\centering
	\subfloat[Pawn movement: A pawn can move to the square directly in front of itself, if that square is clear. A pawn on its starting rank has the option of moving two squares.]{
	%
		\newchessgame
		\setchessboard{showmover=false, smallboard}
		\chessboard[setpieces={pa6, Pc4, Pe2, pg7},
			pgfstyle=straightmove,
			shortenstart=1ex,
			arrow=to,
			markmoves={
			a6-a5,
			c4-c5,
			e2-e3,
			e2-e4,
			g7-g6,
			g7-g5
			}
		]
		\label{fig: non capture pawn moves}
	%
	}
	\hspace{0.5cm}
	\subfloat[The white pawn at d5 may capture either the black rook at c6 or the black knight at e6, but not the bishop at d6, which blocks the pawn's ability to move directly forward.]{
	%
		\newchessgame
		\setchessboard{showmover=false, smallboard}
		\chessboard[setpieces={Pd5, rc6, bd6, ne6},
			pgfstyle=straightmove,
			shortenstart=0.3ex,
			shortenend=1ex,
			color=green,
			arrow=to,
			markmoves={
				d5-c6,
				d5-e6
			}
		]
		\label{fig: capture pawn moves}
	%
	}
		\vfill
		\hspace{0.5cm}
	\subfloat[\textit{En passant} capture, assuming that the black pawn has just moved from c7 to c5. The white pawn moves to the c6-square and the black pawn is removed.]{
	%
		\newchessgame
		\setchessboard{showmover=false, smallboard}
		\chessboard[setpieces={Pd5, pc5},
			pgfstyle=straightmove,
			shortenend=1ex,
			color=blue,
			arrow=to,
			markmoves={c7-c5},
			color=green,
			shortenstart=0.3ex,
			markmoves={d5-c6}
		]
		\label{fig: en passant capture}
	%
	}
	\captionof{figure}{Pawn movement.}
\end{figure}
%
Move generating for pawns is more difficult compared to other pieces. Pawns are the only piece that can move into one direction.
Pawns can only capture diagonally, one square forward and to the left or right.
Another unusual move is the \textit{en passant} capture.
When an enemy pawns moves forward two squares instead of one, the square behind the pawn that just made this move, becomes a possible move for the enemy pawn. Note that the pawn that moved two squares passed over a square that was attack by an enemy pawn.
That enemy pawn, which would have been able to capture the moving pawn had it advanced only one square, is entitled to capture the moving pawn ``in passing'' as if it had advanced only one square.
The capturing pawn moves into the empty square over which the moving pawn passed, and the moving pawn is removed from the board.
Note that an en passant capture must take place directly when the enemy pawn moves two square forward. 
If a player chose not to play the en passant move, the move cannot be applied to the same pawn when the player has turn again.
Pawns can also promote into other pieces when reaching the opposite side of the board (the first rank of the other player).
When the a pawn reaches this position, it is promoted into another piece of that players's choice: a queen, rook, bishop or knight of the same color. 
Most of the times, players choose to promote their pawn into a queen.
When an other piece is chosen, it is also called ``underpromotion''.

Now we know the possible moves of the pawn, lets dive into the move generation using bitboards.
Let first talk about the pawn pushes, which do not involve captures.
To check whether pawn can move forward one square is easily determined by shifting to the North when the pawn is white, and to the South when the pawn is black. Finally, we intersect the resulting bitboard with the bitboard containing all of empty squares. The code is shown in \Cref{fig: pawn single push}.
%
\begin{listing}
\begin{minted}{c++}
uint64_t whitePawnsSinglePushMoves(uint64_t whitePawns, uint64_t emptySquares){
	return northOne(whitePawns) & emptySquares;
}

uint64_t blackPawnsSinglePushMoves(uint64_t blackPawns, uint64_t emptySquares){
	return southOne(blackPawns) & emptySquares;
}
\end{minted}
\captionof{listing}{Calculation of the single push squares for the pawns.}
\label{fig: pawn single push}
\end{listing}
%
For calculating the double push targets, we shift the single push square in the right direction and check whether the new rank of the pawn is empty (rank 4 for white pawns, rank 5 for black pawns). The code is shown in \Cref{fig: pawns double push}.
%
\begin{listing}
\begin{minted}{c++}
uint64_t whitePawnsDoublePushMoves(uint64_t whitePawns, uint64_t emptySquares){
	// mask in which the bits on rank 4 are turned on.
	const uint64_t rank4 = 0x00000000FF000000;
	uint64_t singlePushs = whitePawnsSinglePushMoves(whitePawns, emptySquares);
	return northOne(singlePushs) & rank4 & emptySquares;
}

uint64_t blackPawnsDoublePushMoves(uint64_t blackPawns, uint64_t emptySquares){
	// mask in which the bits on rank 5 are turned on.
	const uint64_t rank5 = 0x000000FF00000000;
	uint64_t singlePushs = blackPawnsSinglePushMoves(blackPawns, emptySquares);
	return southOne(singlePushs) & rank5 & emptySquares;
}
\end{minted}
\captionof{listing}{Calculation of the double push squares for the pawns.}
\label{fig: pawns double push}
\end{listing}
%
To get the pawns that are able to push one square, we shift the free squares towards the rank closest to the player and intersect the result with all the pawns.
We describe the calculations to get the set of source squares of pawns being able to double push for white pawns:
We start from the rank 4 for white pawns.
We shift the rank, intersected with the empty squares set, towards rank 3. 
Now we intersect it again with the empty squares to verify if rank 3 is empty. Finally, we call \texttt{whitePawnsAbleToPush()} with the white pawns and this calculated empty rank 3. 
For black pawns, the calculation can be done similar for black.
The code snippet is shown in \Cref{fig: white pawns able to push}.
%
\begin{listing}
\begin{minted}{c++}
uint64_t whitePawnsAbleToPush(uint64_t whitePawns, uint64_t emptySquares){
	return southOne(emptySquares) & whitePawns;
}

uint64_t whitePawnsAbleToDoublePush(uint64_t whitePawns, uint64_t emptySquares) {
	const uint64_t rank4 = 0x00000000FF000000;
	uint64_t emptyRank3 = southOne(emptySquares & rank4) & emptySquares;
	return whitePawnsAbleToPush(whitePawns, emptyRank3);
}
\end{minted}
\captionof{listing}{Calculation of the source squares of the white pawns that are able to double push.}
\label{fig: white pawns able to push}
\end{listing}
%
Since double pushing triggers determination of en passant target square, it makes sense to serialize both sets separately for different move encoding. The same applies to promotion moves.



\subsection{Sliding Pieces \& Magic Bitboards}
The sliding pieces in chess are the rook, bishop and the queen.
As these pieces can have multiple rays of squares they can move to, we also have to deal with obstructions blocking certain rays.
Friendly and enemy pieces can block an attack ray in one particular direction.
Although we can use the shifting operations, as we have seen in the calculation of the move set for pawns, knights and the king, for calculating the moves for these pieces, other methods exists as well.
For the sliding pieces, we will use so called magic bitboards.
Before diving into these magic bitboards, lets explain some terminology by showing an example.
Lets assume we are dealing with the chess board shown in \Cref{fig: magic bitboards example board}.
The corresponding bitboard is shown in \Cref{fig: magic example board}.
%
\begin{figure}[H]
	\centering
	\subfloat[Example position.]{
	\newchessgame
	\setchessboard{showmover=false, smallboard}
	\chessboard[
		setpieces={
		%White pieces
		Ra1, Bc1, Pa2, Pb3, Pd2, Pd3, Ke2, Pe3, Pf2, Nf3, Pg4,
		%Black pieces
		pa7, pb7, pc7, nd7, kf7, rg8, rc6, pd6, pe5, pf6
		},
	]
	}
	\hspace{1cm}
	\subfloat[The bitboard of the example position. We also call this the \textbf{occupancy bitboard}.]{
		\begin{tikzpicture}[baseline]
		\def\x{0.53}
		\draw[xstep=\x cm,ystep=\x cm,color=gray] (0,0) grid (\x*8,\x*8);
		\matrix[matrix of nodes,
		% http://tex.stackexchange.com/questions/15093/single-ampersand-used-with-wrong-catcode-error-using-tikz-matrix-in-beamer
		ampersand replacement=\&,
		inner sep=0pt,
		anchor=south west,
		nodes={inner sep=0pt,text width=\x cm,align=center,minimum height=\x cm}
		]{
			0 \& 0 \& 0 \& 0 \& 0 \& 0 \& \textbf{1} \& 0 \\
			\textbf{1} \& \textbf{1} \& \textbf{1} \& \textbf{1} \& 0 \& \textbf{1} \& 0 \& 0 \\
			0 \& 0 \& \textbf{1} \& \textbf{1} \& 0 \& \textbf{1} \& 0 \& 0 \\
			0 \& 0 \& 0 \& 0 \& \textbf{1} \& 0 \& 0 \& 0 \\
			0 \& 0 \& 0 \& 0 \& 0 \& 0 \& \textbf{1} \& 0 \\
			0 \& \textbf{1} \& 0 \& \textbf{1} \& \textbf{1} \& \textbf{1} \& 0 \& 0 \\
			\textbf{1} \& 0 \& 0 \& \textbf{1} \& \textbf{1} \& \textbf{1} \& 0 \& 0 \\
			\textbf{1} \& 0 \& \textbf{1} \& 0 \& 0 \& 0 \& 0 \& 0 \\
		};
		\end{tikzpicture}
		\label{fig: magic example board}
	}
	\vfill
	\subfloat[The mask that removes all the pieces that cannot influence the possible moves of the rook on \texttt{c6}. We call this the \textbf{attack bitboard}.]{
		\begin{tikzpicture}[baseline]
		\def\x{0.53}
		\draw[xstep=\x cm,ystep=\x cm,color=gray] (0,0) grid (\x*8,\x*8);
		\matrix[matrix of nodes,
		% http://tex.stackexchange.com/questions/15093/single-ampersand-used-with-wrong-catcode-error-using-tikz-matrix-in-beamer
		ampersand replacement=\&,
		inner sep=0pt,
		anchor=south west,
		nodes={inner sep=0pt,text width=\x cm,align=center,minimum height=\x cm}
		]{
			0 \& 0 \& \textbf{1} \& 0 \& 0 \& 0 \& 0 \& 0 \\
			0 \& 0 \& \textbf{1} \& 0 \& 0 \& 0 \& 0 \& 0 \\
			\textbf{1} \& \textbf{1} \& 0 \& \textbf{1} \& \textbf{1} \& \textbf{1} \& \textbf{1} \& \textbf{1} \\
			0 \& 0 \& \textbf{1} \& 0 \& 0 \& 0 \& 0 \& 0 \\
			0 \& 0 \& \textbf{1} \& 0 \& 0 \& 0 \& 0 \& 0 \\
			0 \& 0 \& \textbf{1} \& 0 \& 0 \& 0 \& 0 \& 0 \\
			0 \& 0 \& \textbf{1} \& 0 \& 0 \& 0 \& 0 \& 0 \\
			0 \& 0 \& \textbf{1} \& 0 \& 0 \& 0 \& 0 \& 0 \\
		};
		\end{tikzpicture}
		\label{fig: rook c6 attack bitboard}
		}
	\hspace{1cm}
		\subfloat[Result of the bitboard representing the example position AND-ed with the corresponding mask of a rook on \texttt{c6}. We also call this the \textbf{blocker bitboard}.]{
			\begin{tikzpicture}[baseline]
			\def\x{0.53}
			\draw[xstep=\x cm,ystep=\x cm,color=gray] (0,0) grid (\x*8,\x*8);
			\matrix[matrix of nodes,
			% http://tex.stackexchange.com/questions/15093/single-ampersand-used-with-wrong-catcode-error-using-tikz-matrix-in-beamer
			ampersand replacement=\&,
			inner sep=0pt,
			anchor=south west,
			nodes={inner sep=0pt,text width=\x cm,align=center,minimum height=\x cm}
			]{
				0 \& 0 \& 0 \& 0 \& 0 \& 0 \& 0 \& 0 \\
				0 \& 0 \& \textbf{1} \& 0 \& 0 \& 0 \& 0 \& 0 \\
				0 \& 0 \& 0 \& \textbf{1} \& 0 \& \textbf{1} \& 0 \& 0 \\
				0 \& 0 \& 0 \& 0 \& 0 \& 0 \& 0 \& 0 \\
				0 \& 0 \& 0 \& 0 \& 0 \& 0 \& 0 \& 0 \\
				0 \& 0 \& 0 \& 0 \& 0 \& 0 \& 0 \& 0 \\
				0 \& 0 \& 0 \& 0 \& 0 \& 0 \& 0 \& 0 \\
				0 \& 0 \& \textbf{1} \& 0 \& 0 \& 0 \& 0 \& 0 \\
			};
			\end{tikzpicture}
			\label{fig: magic blocker bitboard}
		}
		\captionof{figure}{From chess board to bitboard.}
		\label{fig: magic bitboards example board}
\end{figure}
%
Lets say we want to calculate the moves for the rook on \texttt{c6}. 
As we know, rooks move in the following directions: N, E, S and W. 
In this particular case, the rook on \texttt{c6} does not care about pieces that are not on the same rank or on the same file as where he currently is standing.
Therefore, we use the mask shown in \Cref{fig: rook c6 attack bitboard}.
This particular mask is essentially a bitboard where all the possible moves of the rook on \texttt{c6} are present, as if there were no pieces blocking its attack rays. 
We call such a mask for a piece on a particular square an \textbf{attack bitboard}.
If we \texttt{AND} this \textbf{attack bitboard} with the bitboard representing our example position (\Cref{fig: magic example board}), we get the so called \textbf{blocker bitboard} as shown in \Cref{fig: magic blocker bitboard}. 
Now we want the legal moves for our rook.
A simple algorithm would be to iterate over the squares in all the positions (N, E, W and S) and mark all of the empty squares.
This is done until an occupied square is found.
We then return the marked bits as legal moves.

However, a better method exists.
We have to make two observations:
%
\begin{enumerate}
	\item We do not care about the edges on our board: if there is an enemy piece on an edge, we can always capture it. 
	Therefore, an edge piece does not contribute to the information we need for determining the set of legal moves.
	
	\item So what is the maximum number of blocker bitboards when we consider all of the squares on the chess board?
	Well, lets place a rook on \texttt{a1}. 
	As we said previously, we do not count the edge squares as blockers. 
	In the case of a rook standing on \texttt{a1}, the edges are: \texttt{a8} and \texttt{h1}. 
	This leaves us with 6 positions to the north which could be occupied (\texttt{a2} through \texttt{a7}) and 6 positions to the east that could be occupied (\texttt{b1} through \texttt{g1}).
	Note that for other positions, the squares that could be occupied would be less or equal to 12.
	Thus, there are at maximum $2^{12}$ blocker bitboards for a random square occupied by a rook. 
	This number is not big at all: if we would have a lookup table, we can simply look up the move set for each of the $2^{12}$ states.
\end{enumerate}
%
The next question that arises is: how do we go from a blocker bitboard to an index in this lookup table? 
This is where magic bitboards come in. The idea is as follows. 
We find a \textbf{magic number} $m_i$ for every square such that multiplying the blocker bitboard by $m_i$ gives a perfect hash into the indices of the lookup table.
A magic number for a rook on \texttt{c6} is:
%
\begin{align*}
m_i &= 432627108460691524
\end{align*}
%
Lets review our blocker bitboard as shown \Cref{fig: magic blocker bitboard}. 
As a 64-bit number, this is simply $b = 1169880371953668b=1169880371953668$.
Now we calculate
%
\begin{align*}
mb &= 4692364060652732688 \pmod{64}
\end{align*}
%
Because there are only ten relevant occupancy bits for a rook on \texttt{c6}, we only retrieve the top \textbf{10} bits of this hash. This gives the following index:
%
\begin{align*}
\text{idx} &= mb >> 54 = 2083
\end{align*}
%
Now we can use this index to retrieve the valid moves for a rook on \texttt{c6} with blocker bitboard shown in \Cref{fig: magic blocker bitboard}.
Simple right?
But how do we find these magic numbers and what do we need to take into account when finding such a hash function?
Well, lets again take a look at the blocker bitboard shown in \Cref{fig: magic blocker bitboard}.
We can easily see that the move set for this blocker board is equal to the moves shown in \Cref{fig: move set from blocker bitboard}.
%
\begin{figure}
	\centering		
	\begin{tikzpicture}[baseline]
	\def\x{0.53}
	\draw[xstep=\x cm,ystep=\x cm,color=gray] (0,0) grid (\x*8,\x*8);
	\matrix[matrix of nodes,
	% http://tex.stackexchange.com/questions/15093/single-ampersand-used-with-wrong-catcode-error-using-tikz-matrix-in-beamer
	ampersand replacement=\&,
	inner sep=0pt,
	anchor=south west,
	nodes={inner sep=0pt,text width=\x cm,align=center,minimum height=\x cm}
	]{
		0 \& 0 \& 0 \& 0 \& 0 \& 0 \& 0 \& 0 \\
		0 \& 0 \& \textbf{1} \& 0 \& 0 \& 0 \& 0 \& 0 \\
		\textbf{1} \& \textbf{1} \& 0 \& \textbf{1} \& 0 \& 0 \& 0 \& 0 \\
		0 \& 0 \& \textbf{1} \& 0 \& 0 \& 0 \& 0 \& 0 \\
		0 \& 0 \& \textbf{1} \& 0 \& 0 \& 0 \& 0 \& 0 \\
		0 \& 0 \& \textbf{1} \& 0 \& 0 \& 0 \& 0 \& 0 \\
		0 \& 0 \& \textbf{1} \& 0 \& 0 \& 0 \& 0 \& 0 \\
		0 \& 0 \& \textbf{1} \& 0 \& 0 \& 0 \& 0 \& 0 \\
	};
	\end{tikzpicture}
	\captionof{figure}{Move set corresonding to the blocker bitboard shown in \Cref{fig: magic blocker bitboard}.}
	\label{fig: move set from blocker bitboard}
\end{figure}
%
But is the blocker bitboard as shown in \Cref{fig: magic blocker bitboard} the only blocker bitboard that maps to the move set in \Cref{fig: move set from blocker bitboard}?
No, it is not!
As we can see in \Cref{fig: magic bitboards example board}, multiple blocker bits are set to the east (E) of the rook on \texttt{c6}. 
But does it matter if there is another blocker bit if there was already a blocker bit before that one (closer to the piece)? 
Well, only blocker bits that are the first to block the ray matter.
Therefore, all of these blocker bitboard that share the same `first' blocker bits map to the same move set.
The reduced blocker bitboard for \Cref{fig: magic blocker bitboard} can be seen in \Cref{fig: reduced blocker bitboard}.
%
\begin{figure}[H]
	\centering
	\begin{tikzpicture}[baseline]
		\def\x{0.53}
		\draw[xstep=\x cm,ystep=\x cm,color=gray] (0,0) grid (\x*8,\x*8);
		\matrix[matrix of nodes,
		% http://tex.stackexchange.com/questions/15093/single-ampersand-used-with-wrong-catcode-error-using-tikz-matrix-in-beamer
		ampersand replacement=\&,
		inner sep=0pt,
		anchor=south west,
		nodes={inner sep=0pt,text width=\x cm,align=center,minimum height=\x cm}
		]{
			0 \& 0 \& 0 \& 0 \& 0 \& 0 \& 0 \& 0 \\
			0 \& 0 \& \textbf{1} \& 0 \& 0 \& 0 \& 0 \& 0 \\
			0 \& 0 \& 0 \& \textbf{1} \& 0 \& 0 \& 0 \& 0 \\
			0 \& 0 \& 0 \& 0 \& 0 \& 0 \& 0 \& 0 \\
			0 \& 0 \& 0 \& 0 \& 0 \& 0 \& 0 \& 0 \\
			0 \& 0 \& 0 \& 0 \& 0 \& 0 \& 0 \& 0 \\
			0 \& 0 \& 0 \& 0 \& 0 \& 0 \& 0 \& 0 \\
			0 \& 0 \& \textbf{1} \& 0 \& 0 \& 0 \& 0 \& 0 \\
		};
	\end{tikzpicture}
	\captionof{figure}{Reduced blocker bitboard for the blocker bitboards shown in \Cref{fig: magic blocker bitboard}. The blockers are as follows: \texttt{c7}, \texttt{d6} and \texttt{c1}.}
	\label{fig: reduced blocker bitboard}
\end{figure}
%
In \Cref{fig: blocker bitboard variations} we can see some variations of the blocker bitboards shown \Cref{fig: magic blocker bitboard} that all can be reduced to the bitboard shown in \Cref{fig: reduced blocker bitboard}.
%
\begin{figure}[H]
\centering
		\subfloat[Blocker bitboard variation 1.]{
			\begin{tikzpicture}[baseline]
			\def\x{0.53}
			\draw[xstep=\x cm,ystep=\x cm,color=gray] (0,0) grid (\x*8,\x*8);
			\matrix[matrix of nodes,
			% http://tex.stackexchange.com/questions/15093/single-ampersand-used-with-wrong-catcode-error-using-tikz-matrix-in-beamer
			ampersand replacement=\&,
			inner sep=0pt,
			anchor=south west,
			nodes={inner sep=0pt,text width=\x cm,align=center,minimum height=\x cm}
			]{
				0 \& 0 \& 0 \& 0 \& 0 \& 0 \& 0 \& 0 \\
				0 \& 0 \& \textbf{1} \& 0 \& 0 \& 0 \& 0 \& 0 \\
				0 \& 0 \& 0 \& \textbf{1} \& \textbf{1} \& \textbf{1} \& 0 \& 0 \\
				0 \& 0 \& 0 \& 0 \& 0 \& 0 \& 0 \& 0 \\
				0 \& 0 \& 0 \& 0 \& 0 \& 0 \& 0 \& 0 \\
				0 \& 0 \& 0 \& 0 \& 0 \& 0 \& 0 \& 0 \\
				0 \& 0 \& 0 \& 0 \& 0 \& 0 \& 0 \& 0 \\
				0 \& 0 \& \textbf{1} \& 0 \& 0 \& 0 \& 0 \& 0 \\
			};
			\end{tikzpicture}
		}
		\hspace{1cm}
			\subfloat[Blocker bitboard variation 2.]{
				\begin{tikzpicture}[baseline]
				\def\x{0.53}
				\draw[xstep=\x cm,ystep=\x cm,color=gray] (0,0) grid (\x*8,\x*8);
				\matrix[matrix of nodes,
				% http://tex.stackexchange.com/questions/15093/single-ampersand-used-with-wrong-catcode-error-using-tikz-matrix-in-beamer
				ampersand replacement=\&,
				inner sep=0pt,
				anchor=south west,
				nodes={inner sep=0pt,text width=\x cm,align=center,minimum height=\x cm}
				]{
					0 \& 0 \& \textbf{1} \& 0 \& 0 \& 0 \& 0 \& 0 \\
					0 \& 0 \& \textbf{1} \& 0 \& 0 \& 0 \& 0 \& 0 \\
					0 \& 0 \& 0 \& \textbf{1} \& 0 \& 0 \& 0 \& \textbf{1} \\
					0 \& 0 \& 0 \& 0 \& 0 \& 0 \& 0 \& 0 \\
					0 \& 0 \& 0 \& 0 \& 0 \& 0 \& 0 \& 0 \\
					0 \& 0 \& 0 \& 0 \& 0 \& 0 \& 0 \& 0 \\
					0 \& 0 \& 0 \& 0 \& 0 \& 0 \& 0 \& 0 \\
					0 \& 0 \& \textbf{1} \& 0 \& 0 \& 0 \& 0 \& 0 \\
				};
				\end{tikzpicture}
			}
			\captionof{figure}{Variations of the blocker bitboard as shown \Cref{fig: magic blocker bitboard}. All of these variations have the same reduced blocker bitboard.}	
			\label{fig: blocker bitboard variations}
\end{figure}
%
Now that we know that not all of the bits matter in finding the move set for a certain blocker bitboard, we continue our journey to find these so-called magic numbers.
Well, lets first calculate all the possible blocker bitboards for the piece on a specific square.
For each blocker bitboard variation, we calculate the corresponding reduced bitboard.
We also maintain a database which we can access to see which index maps to which reduced bitboard.
We then multiply the blocker bitboard variation by a random bitboard (which could be a possible magic number as we can see later).
Now we obtain the index by shifting the result of the multiplication by $n$ bits to the right, where $n$ is the number of bits set in the corresponding attack set of the piece on that square (make sure that the corresponding edge squares are not set!).
Now we access the database using the calculated index.
If the index is not 0, we need to verify if the stored move set is equal to the move set that we would expect for the given blocker bitboard variation.
If the move sets are the same, our mapping is still valid because it will map variations of the same blocker bitboard to the same index.
We continue this validation process for all the different blocker bitboards to see whether our potential magic number maps the right blocker bitboard variations for a given square to the right index.
If no clash occurs during our search we found a magic number for a square!

So in general, generating moves using magic bitboards consists of four steps:
%
\begin{enumerate}
	\item Pick some random 64-bit number, call it \textbf{magic}.
	Get all the possible the blocker bitboard variations of the piece.
	\item Create a database of size $2^{\text{bits}}$ (where bits in the number of bits set in the original attack bitboard).
	\item For each blocker bitboard variation:
	%
	\begin{enumerate}
		\item Calculate the corresponding \textbf{move set} for the variation. 

		\item Create an index by multiplying variation with magic and shift it right with $64 - \text{bits}$.
		\item \textbf{If} database[index] is not 0 \textbf{and} the move set stored at the index in our mapping bitboard is \textbf{not} equal to the move set we expected it to have, then we have a clash. If this happens, we need to clear our mapping and generate a new number for which we want to check whether it is a magic number.
		 
	\item If the move sets are equal or there was no move set stored at this index, we store the move set for the current variation at \texttt{database[index]} and continue.
	\end{enumerate}
	%
	\item If the loop ends with no clashes, then magic is a valid magic number for the piece on the square we were currently looking at.
\end{enumerate}
%