\section{Common Bitboards}
To actually represent all the positions of each piece type and of each color on the chessboard, we need at least 12 bitboards. 
These bitboards are as follows:	\textbf{WhitePawns}, \textbf{WhiteRooks}, \textbf{WhiteKnights}, \textbf{WhiteBishops}, \textbf{WhiteQueens}, \textbf{WhiteKing}, \textbf{BlackPawns}, \textbf{BlackRooks}, \textbf{BlackKnights}, \textbf{BlackBishops}, \textbf{BlackQueens} and\textbf{BlackKing}.
I will only describe the bitboards for the white pieces.
Once you get an understanding of it, you will be able to make bitboards for the initial black pieces yourself.

\subsection{White Pawns}
%
\begin{figure}[H]
	\centering
	\subfloat[Our chess board.]{
		\setchessboard{showmover=false, clearboard, smallboard}
		\def\initialpawns{Pa2, Pb2, Pc2, Pd2, Pe2, Pf2, Pg2, Ph2}
		\chessboard[setpieces=\initialpawns]
	}
	\subfloat[Bitboard representing intial white pawns.]{
		% http://tex.stackexchange.com/questions/12856/tikz-finite-grid-with-character-in-each-cell
		\begin{tikzpicture}[baseline]
		\def\x{0.53}
		\draw[xstep=\x cm,ystep=\x cm,color=gray] (0,0) grid (\x*8,\x*8);
		\matrix[matrix of nodes,
		% http://tex.stackexchange.com/questions/15093/single-ampersand-used-with-wrong-catcode-error-using-tikz-matrix-in-beamer
		ampersand replacement=\&,
		inner sep=0pt,
		anchor=south west,
		nodes={inner sep=0pt,text width=\x cm,align=center,minimum height=\x cm}
		]{
			0 \& 0 \& 0 \& 0 \& 0 \& 0 \& 0 \& 0 \\
			0 \& 0 \& 0 \& 0 \& 0 \& 0 \& 0 \& 0 \\
			0 \& 0 \& 0 \& 0 \& 0 \& 0 \& 0 \& 0 \\
			0 \& 0 \& 0 \& 0 \& 0 \& 0 \& 0 \& 0 \\
			0 \& 0 \& 0 \& 0 \& 0 \& 0 \& 0 \& 0 \\
			0 \& 0 \& 0 \& 0 \& 0 \& 0 \& 0 \& 0 \\
			\textbf{1} \& \textbf{1} \& \textbf{1} \& \textbf{1} \& 
			\textbf{1} \& \textbf{1} \& \textbf{1} \& \textbf{1} \\
			0 \& 0 \& 0 \& 0 \& 0 \& 0 \& 0 \& 0 \\
		};
		\end{tikzpicture}
	}
	\captionof{figure}{The bitboard of the initial white pawns.}
\end{figure}
%
This gives the following bitboard for all the initial white pawns:
%
\begin{center}
	\texttt{0000000011111111000000000000000000000000000000000000000000000000}
\end{center}
%

\subsection{White Rooks}
%
\begin{figure}[H]
	\centering
	\subfloat[Our chess board.]{
		\setchessboard{showmover=false, clearboard, smallboard}
		\chessboard[setpieces={Ra1, Rh1}]
	}
	\subfloat[Bitboard representing intial white rooks.]{
		% http://tex.stackexchange.com/questions/12856/tikz-finite-grid-with-character-in-each-cell
		\begin{tikzpicture}[baseline]
		\def\x{0.53}
		\draw[xstep=\x cm,ystep=\x cm,color=gray] (0,0) grid (\x*8,\x*8);
		\matrix[matrix of nodes,
		% http://tex.stackexchange.com/questions/15093/single-ampersand-used-with-wrong-catcode-error-using-tikz-matrix-in-beamer
		ampersand replacement=\&,
		inner sep=0pt,
		anchor=south west,
		nodes={inner sep=0pt,text width=\x cm,align=center,minimum height=\x cm}
		]{
			0 \& 0 \& 0 \& 0 \& 0 \& 0 \& 0 \& 0 \\
			0 \& 0 \& 0 \& 0 \& 0 \& 0 \& 0 \& 0 \\
			0 \& 0 \& 0 \& 0 \& 0 \& 0 \& 0 \& 0 \\
			0 \& 0 \& 0 \& 0 \& 0 \& 0 \& 0 \& 0 \\
			0 \& 0 \& 0 \& 0 \& 0 \& 0 \& 0 \& 0 \\
			0 \& 0 \& 0 \& 0 \& 0 \& 0 \& 0 \& 0 \\
			0 \& 0 \& 0 \& 0 \& 0 \& 0 \& 0 \& 0 \\
			\textbf{1} \& 0 \& 0 \& 0 \& 0 \& 0 \& 0 \& \textbf{1} \\
		};
		\end{tikzpicture}
	}
	\captionof{figure}{The bitboard of the initial white rooks.}
\end{figure}
%
This gives the following bitboard for all the initial white bishops:
%
\begin{center}
	\texttt{1000000100000000000000000000000000000000000000000000000000000000}
\end{center}
%

\subsection{White Knights}
%
\begin{figure}[H]
	\centering
	\subfloat[Our chess board.]{
		\setchessboard{showmover=false, clearboard, smallboard}
		\chessboard[setpieces={Nb1, Ng1}]
	}
	\subfloat[Bitboard representing intial white knights.]{
		% http://tex.stackexchange.com/questions/12856/tikz-finite-grid-with-character-in-each-cell
		\begin{tikzpicture}[baseline]
		\def\x{0.53}
		\draw[xstep=\x cm,ystep=\x cm,color=gray] (0,0) grid (\x*8,\x*8);
		\matrix[matrix of nodes,
		% http://tex.stackexchange.com/questions/15093/single-ampersand-used-with-wrong-catcode-error-using-tikz-matrix-in-beamer
		ampersand replacement=\&,
		inner sep=0pt,
		anchor=south west,
		nodes={inner sep=0pt,text width=\x cm,align=center,minimum height=\x cm}
		]{
			0 \& 0 \& 0 \& 0 \& 0 \& 0 \& 0 \& 0 \\
			0 \& 0 \& 0 \& 0 \& 0 \& 0 \& 0 \& 0 \\
			0 \& 0 \& 0 \& 0 \& 0 \& 0 \& 0 \& 0 \\
			0 \& 0 \& 0 \& 0 \& 0 \& 0 \& 0 \& 0 \\
			0 \& 0 \& 0 \& 0 \& 0 \& 0 \& 0 \& 0 \\
			0 \& 0 \& 0 \& 0 \& 0 \& 0 \& 0 \& 0 \\
			0 \& 0 \& 0 \& 0 \& 0 \& 0 \& 0 \& 0 \\
			0 \& \textbf{1} \& 0 \& 0 \& 0 \& 0 \& \textbf{1} \& 0 \\
		};
		\end{tikzpicture}
	}
	\captionof{figure}{The bitboard of the initial white knights.}
\end{figure}
%
This gives the following bitboard for all the initial white knights:
%
\begin{center}
	\texttt{0100001000000000000000000000000000000000000000000000000000000000}
\end{center}
%

\subsection{White Bishops}
%
\begin{figure}[H]
	\centering
	\subfloat[Our chess board.]{
		\setchessboard{showmover=false, clearboard, smallboard}
		\chessboard[setpieces={Bc1, Bf1}]
	}
	\subfloat[Bitboard representing intial white pawns.]{
		% http://tex.stackexchange.com/questions/12856/tikz-finite-grid-with-character-in-each-cell
		\begin{tikzpicture}[baseline]
		\def\x{0.53}
		\draw[xstep=\x cm,ystep=\x cm,color=gray] (0,0) grid (\x*8,\x*8);
		\matrix[matrix of nodes,
		% http://tex.stackexchange.com/questions/15093/single-ampersand-used-with-wrong-catcode-error-using-tikz-matrix-in-beamer
		ampersand replacement=\&,
		inner sep=0pt,
		anchor=south west,
		nodes={inner sep=0pt,text width=\x cm,align=center,minimum height=\x cm}
		]{
			0 \& 0 \& 0 \& 0 \& 0 \& 0 \& 0 \& 0 \\
			0 \& 0 \& 0 \& 0 \& 0 \& 0 \& 0 \& 0 \\
			0 \& 0 \& 0 \& 0 \& 0 \& 0 \& 0 \& 0 \\
			0 \& 0 \& 0 \& 0 \& 0 \& 0 \& 0 \& 0 \\
			0 \& 0 \& 0 \& 0 \& 0 \& 0 \& 0 \& 0 \\
			0 \& 0 \& 0 \& 0 \& 0 \& 0 \& 0 \& 0 \\
			0 \& 0 \& 0 \& 0 \& 0 \& 0 \& 0 \& 0 \\
			0 \& 0 \& \textbf{1} \& 0 \& 0 \& \textbf{1} \& 0 \& 0 \\
		};
		\end{tikzpicture}
	}
	\captionof{figure}{The bitboard of the initial white rooks.}
\end{figure}
%
This gives the following bitboard for all the initial white bishops:
%
\begin{center}
	\texttt{0010010000000000000000000000000000000000000000000000000000000000}
\end{center}
%

\subsection{White Queens}
%
\begin{figure}[H]
	\centering
	\subfloat[Our chess board.]{
		\setchessboard{showmover=false, clearboard, smallboard}
		\chessboard[setpieces={Qd1}]
	}
	\subfloat[Bitboard representing intial white queen.]{
		% http://tex.stackexchange.com/questions/12856/tikz-finite-grid-with-character-in-each-cell
		\begin{tikzpicture}[baseline]
		\def\x{0.53}
		\draw[xstep=\x cm,ystep=\x cm,color=gray] (0,0) grid (\x*8,\x*8);
		\matrix[matrix of nodes,
		% http://tex.stackexchange.com/questions/15093/single-ampersand-used-with-wrong-catcode-error-using-tikz-matrix-in-beamer
		ampersand replacement=\&,
		inner sep=0pt,
		anchor=south west,
		nodes={inner sep=0pt,text width=\x cm,align=center,minimum height=\x cm}
		]{
			0 \& 0 \& 0 \& 0 \& 0 \& 0 \& 0 \& 0 \\
			0 \& 0 \& 0 \& 0 \& 0 \& 0 \& 0 \& 0 \\
			0 \& 0 \& 0 \& 0 \& 0 \& 0 \& 0 \& 0 \\
			0 \& 0 \& 0 \& 0 \& 0 \& 0 \& 0 \& 0 \\
			0 \& 0 \& 0 \& 0 \& 0 \& 0 \& 0 \& 0 \\
			0 \& 0 \& 0 \& 0 \& 0 \& 0 \& 0 \& 0 \\
			0 \& 0 \& 0 \& 0 \& 0 \& 0 \& 0 \& 0 \\
			0 \& 0 \& 0 \& \textbf{1} \& 0 \& 0 \& 0 \& 0 \\
		};
		\end{tikzpicture}
	}
	\captionof{figure}{The bitboard of the initial white queen.}
\end{figure}
%
This gives the following bitboard for the initial white queen:
%
\begin{center}
	\texttt{0001000000000000000000000000000000000000000000000000000000000000}
\end{center}
%

\subsection{White King}
%
\begin{figure}[H]
	\centering
	\subfloat[Our chess board.]{
	\newchessgame
		\setchessboard{showmover=false, clearboard, smallboard}
		\chessboard[setpieces={Ke1}]
	}
	\subfloat[Bitboard representing intial white king.]{
		% http://tex.stackexchange.com/questions/12856/tikz-finite-grid-with-character-in-each-cell
		\begin{tikzpicture}[baseline]
		\def\x{0.53}
		\draw[xstep=\x cm,ystep=\x cm,color=gray] (0,0) grid (\x*8,\x*8);
		\matrix[matrix of nodes,
		% http://tex.stackexchange.com/questions/15093/single-ampersand-used-with-wrong-catcode-error-using-tikz-matrix-in-beamer
		ampersand replacement=\&,
		inner sep=0pt,
		anchor=south west,
		nodes={inner sep=0pt,text width=\x cm,align=center,minimum height=\x cm}
		]{
			0 \& 0 \& 0 \& 0 \& 0 \& 0 \& 0 \& 0 \\
			0 \& 0 \& 0 \& 0 \& 0 \& 0 \& 0 \& 0 \\
			0 \& 0 \& 0 \& 0 \& 0 \& 0 \& 0 \& 0 \\
			0 \& 0 \& 0 \& 0 \& 0 \& 0 \& 0 \& 0 \\
			0 \& 0 \& 0 \& 0 \& 0 \& 0 \& 0 \& 0 \\
			0 \& 0 \& 0 \& 0 \& 0 \& 0 \& 0 \& 0 \\
			0 \& 0 \& 0 \& 0 \& 0 \& 0 \& 0 \& 0 \\
			0 \& 0 \& 0 \& 0 \& \textbf{1} \& 0 \& 0 \& 0 \\
		};
		\end{tikzpicture}
	}
	\captionof{figure}{The bitboard for the initial white king.}
\end{figure}
%
This gives the following bitboard for the initial white king:
%
\begin{center}
	\texttt{0000100000000000000000000000000000000000000000000000000000000000}
\end{center}
%